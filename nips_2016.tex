\documentclass{article}

% if you need to pass options to natbib, use, e.g.:
% \PassOptionsToPackage{numbers, compress}{natbib}
% before loading nips_2016
%
% to avoid loading the natbib package, add option nonatbib:
% \usepackage[nonatbib]{nips_2016}

\usepackage{nips_2016}

% to compile a camera-ready version, add the [final] option, e.g.:
% \usepackage[final]{nips_2016}

\usepackage[utf8]{inputenc} % allow utf-8 input
\usepackage[T1]{fontenc}    % use 8-bit T1 fonts
\usepackage{hyperref}       % hyperlinks
\usepackage{url}            % simple URL typesetting
\usepackage{booktabs}       % professional-quality tables
\usepackage{amsfonts}       % blackboard math symbols
\usepackage{nicefrac}       % compact symbols for 1/2, etc.
\usepackage{microtype}      % microtypography
\usepackage[]{algorithm2e}

\newtheorem{theorem}{Theorem}
\newtheorem{definition}{Definition}

\title{LSHTS:~A Fast Time Series Similarity Search Framework using Locality Sensitive Hashing}

% The \author macro works with any number of authors. There are two
% commands used to separate the names and addresses of multiple
% authors: \And and \AND.
%
% Using \And between authors leaves it to LaTeX to determine where to
% break the lines. Using \AND forces a line break at that point. So,
% if LaTeX puts 3 of 4 authors names on the first line, and the last
% on the second line, try using \AND instead of \And before the third
% author name.

\author{
  Chen~Luo\thanks{Use footnote for providing further
    information about author (webpage, alternative
    address)---\emph{not} for acknowledging funding agencies.} \\
  Department of Computer Science\\
  Rice University\\
  Houston, Texas \\
  \texttt{cl67@rice.edu} \\
  %% examples of more authors
  \And
  Anshumali Shrivastava \\
  Rice University \\
  Houston, Texas \\
  \texttt{anshumali@rice.edu} \\
  %% \AND
  %% Coauthor \\
  %% Affiliation \\
  %% Address \\
  %% \texttt{email} \\
  %% \And
  %% Coauthor \\
  %% Affiliation \\
  %% Address \\
  %% \texttt{email} \\
  %% \And
  %% Coauthor \\
  %% Affiliation \\
  %% Address \\
  %% \texttt{email} \\
}

\begin{document}
% \nipsfinalcopy is no longer used

\maketitle

\begin{abstract}
  Abstract!
\end{abstract}

\section{Introduction}

The focus of this paper will be on the time series data. A time series is defined as a sequence of values $\{x_1,x_2,...,x_m\}$
associated with timestamps $\{t(x_1), t(x_2),..., t(x_m)\}$ that typically have the relationship of $t(x_i) = t(x_{i-1})+\tau$, where $\tau$ is  the sampling interval, and $m$ is the number of points in the time series. Time series mining is ubiquitous in data driven applications including robotics, medicine~\cite{oates2000method,caracca2000discovering}, speech~\cite{rabiner1993fundamentals}, object detection in vision~\cite{yang2002detecting, sonka2014image}, High Performance Computing (HPC) and system failure diagnosis~\cite{luo2014correlating,sun2014querying}, Earth Science \cite{mudelsee2013climate}, and Finance \cite{granger2014forecasting}, etc.

Second paragraph introduce the importance of fast nearest neighbor search. And the limitation to handle the large scale data set.

Third paragraph introduce the related works to handle the fast nearest neighbor search and the corresponding limitation. 

Forth paragraph introduce our proposal. The overall introduction of our framework. 

Introduce the main contribution of our paper.
1. We propose a new time series similarity search framework using locality sensitive hashing schema.

2. The experimental result shows the effectiveness and efficiency of our framework.

\section{Background}
Show the background of our method.

\subsection{Notation and Definition}

In this section, we introduce some notions of time series. 

A time series is defined as follow:
\begin{definition}[Time Series]
A time series, denoted as $X = (x_1,x_2,...,x_m)$, where $m$ is the number of points in the time series. The timestamps of a time series, denoted as $TX = (t(x1), t(x2),..., t(xn))$, have the relationship of $t(x_i) = t(x_{i-1})+\tau$, where $\tau$is the sampling interval.
\end{definition}

In this paper, we interested in the problem of time series similarity search problem:
\begin{definition}[Top $1$ Similarity Search]
Given a data set $\{X_i|0<i<N-1\}$, where $N$ denotes the size of this data set. Given a query time series $X_q$, the top $1$ similarity search is to find the a time series $X^* \in D$, where

\[
X^* = \arg \max_{X \in D} S(X_q,X)
\]
where $S(X,Y)$ denote the similarity between time series $X$ and $Y$. We introduce the similarity measures in Section \ref{relatedwork}. In this work, we are more interested in the problem of top-k similarity search problem, which is finding the top $k$ most similar time series to $X_q$ in $D$. It is pointed out that, the size $N$ of data set $D$ often huge (i.e. million level). And in this paper, our aim is to do fast searching top $k$ time series.

\end{definition}

\subsection{Related Works}
\label{relatedwork}

First paragraph introduce the similarity between time series data.
Dynamic time warping \cite{rakthanmanon2012searching} time series is regarded as the most widely used similarity method between time series. 

Second paragraph introduce the some top-k similarity search related works in this area.

Third paragraph introduce the LSH methods.


\section{LSHTS Framework}
In this section, we introduce the LSHTS framework in details. Our framework two steps. First step is to transfer a time series into a set using sliding and n-gram.
Then using weghted min-hash to do fast nearest neighbor search.

\subsection {Time Series sliding}

\subsection {N-gram}

\subsection {Weighted Minhash}

\begin{algorithm}[h]
	\KwData{Vector $x$, and random seed[][]}
	\KwResult{Hash Pairs}
	initialization $i=0$\;
	\For{$i$ to k}{
		\For{Iteration over $x_i$}
		{
			random seeds = seed[i][j]\;
			sample $r_{i,j}$ and $c_{i,j}$ \~ Gamma(2,1)\;
			sample $\beta_{i,j}$ \~ Gamma(0,1)\;
			$t_j=|\frac{\log x_i}{r_{i,j}}+\beta_{i,j}|$\;
			$y_i=exp(r_{i,j}(t_j-beta_{i,j}))$\;
			$z_j=y_i*exp(r_{i,j})$\;
			$a_j=c_{i,j}/z_j$
			
		}
		$K^* = \arg\min_ja_i$\;
		HashPairs[i]=($k^*,t_k^*$)
	}
	\caption{Weighted Minhash (CWS)}
\end{algorithm}

For more details of this algorithm, please refer to paper. \cite{ioffe2010improved}. 

\subsection{Overall Framework}

\section{Evaluation}

\subsection{Dataset}

%\begin{figure}[h]
%  \centering
%  \fbox{\rule[-.5cm]{0cm}{4cm} \rule[-.5cm]{4cm}{0cm}}
%  \caption{Sample figure caption.}
%\end{figure}
%
%\subsection{Tables}
%
%\begin{table}[t]
%  \caption{Sample table title}
%  \label{sample-table}
%  \centering
%  \begin{tabular}{lll}
%    \toprule
%    \multicolumn{2}{c}{Part}                   \\
%    \cmidrule{1-2}
%    Name     & Description     & Size ($\mu$m) \\
%    \midrule
%    Dendrite & Input terminal  & $\sim$100     \\
%    Axon     & Output terminal & $\sim$10      \\
%    Soma     & Cell body       & up to $10^6$  \\
%    \bottomrule
%  \end{tabular}
%\end{table}

\subsubsection*{Acknowledgments}

We are thankful for...

\section*{References}
\small

\bibliographystyle{abbrv}
\bibliography{nips_2016}

%[1] Alexander, J.A.\ \& Mozer, M.C.\ (1995) Template-based algorithms
%for connectionist rule extraction. In G.\ Tesauro, D.S.\ Touretzky and
%T.K.\ Leen (eds.), {\it Advances in Neural Information Processing
%  Systems 7}, pp.\ 609--616. Cambridge, MA: MIT Press.
%
%[2] Bower, J.M.\ \& Beeman, D.\ (1995) {\it The Book of GENESIS:
%  Exploring Realistic Neural Models with the GEneral NEural SImulation
%  System.}  New York: TELOS/Springer--Verlag.
%
%[3] Hasselmo, M.E., Schnell, E.\ \& Barkai, E.\ (1995) Dynamics of
%learning and recall at excitatory recurrent synapses and cholinergic
%modulation in rat hippocampal region CA3. {\it Journal of
%  Neuroscience} {\bf 15}(7):5249-5262.

\end{document}
